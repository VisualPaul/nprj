\documentclass{beamer}

 \usepackage{beamerthemesplit} % // Activate for custom appearance
 \usecolortheme[snowy]{owl}
 

\title{Estimating results of linear elasticity PDE using Gaussian process regression}
\author{Pavel Shevchuk}
\date{26 March 2021}

\begin{document}

\frame{\titlepage}

\section {Overview}
\frame
{
  \frametitle{Motivation}
  In engineering tasks we often need to repeatedly re-evaluate PDEs while varying their parameters.
  
  Is it possible to reduce computational cost by using machine learning methods to estimate the solution instead of evaluating it?
}

\section{PDE solution}
%\subsection{Adversarial attacks}
\frame
{
 \frametitle{Equation}
  A partial derivative equation describing linear elasticity was chosen: 
  
  
\begin{align*}
-\nabla\cdot\sigma &= f\hbox{ in }\Omega\\ 
\sigma &= \lambda\,\hbox{tr}\,(\varepsilon) I + 2\mu\varepsilon\\ 
\varepsilon &= \frac{1}{2}\left(\nabla u + (\nabla u)^{\top}\right)
\end{align*}
where $\mu$ and $\lambda$ are the material elasticity parameters.
}

%\subsection{Adversarial attacks}
\frame
{
 \frametitle{Parameters}
  The deflection of the beam under its own weight was considered.  The following parameters were varied:
  \begin{itemize}
  \item $\mu$, $\lambda$ - Lomé's parameters, between 0.1 and 5.
  \item $\rho$ - beam density, between 0.1 and 5.
  \item $g$ - free fall acceleartion, between 0.004 and 0.036
  \end{itemize}
  
  The length of a beam was set to be 1 and the other dimensions are 0.2.
  
  The resulting problem was solved using the finite element method.
  
  As a target value, the maximum displacement of the material was used.
}

\section{Regression and experiments}

\frame
{
   \frametitle{Overall setup}
   For prediction, Gaussian Process Regression was used with RBF kernel and $N(0,1)$ prior.
   
   To evaluate the performance, 100 points were chosen. Three values were reported for each point:
   \begin{itemize}
   \item absolute difference between the GPR prediction and the FEM-calculated value
   \item mean standard deviation of the GPR prediction on these points
   \item logarithm of likelihood of the true values for these points
   \end{itemize}
   
   Two strategies for choosing training points were evaluated: picking parameters randomly and uniformly, and picking 10 parameters randomly and uniformly, but only evaluating FEM on the worst of them in terms of predicted standard deviation.
}

\frame
{
  \frametitle{Average error}
  \includegraphics[scale=0.3]{prj_ae}
  }
\frame
{
  \frametitle{Average standard deviation}
  \includegraphics[scale=0.3]{prj_std}
  }
\frame
{
  \frametitle{Log likelihood}
  \includegraphics[scale=0.3]{prj_ll}
  }

\frame
{
\frametitle{Results discussion}
It is possible to estimate using machine learning methods, functions that are too costly to compute.

Using GPR to create samples for GPR breaks it in practice as well as in theory.

GPR is capable of finding your mesh, you have to be cautious.

}
\end{document}